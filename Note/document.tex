\documentclass{article}%book,report,letter
\usepackage{ctex}
\usepackage{graphicx}
\usepackage{amsmath}
\graphicspath{{figures/}}

\begin{document}
	\section{sde解的存在唯一性}给定T令$b(\cdot, \cdot):[0, T] \times \mathbf{R}^{n} \rightarrow \mathbf{R}^{n}, \sigma(\cdot, \cdot):[0, T] \times \mathbf{R}^{n} \rightarrow \mathbf{R}^{n \times m}$是可测函数满足
		\begin{equation}
		|b(t, x)|+|\sigma(t, x)| \leq C(1+|x|) ; \quad x \in \mathbf{R}^{n}, \quad t \in[0, T]
		\end{equation}
		\begin{equation}
		|b(t, x)-b(t, y)|+|\sigma(t, x)-\sigma(t, y)| \leq D|x-y| ; \quad x, y \in \mathbf{R}^{n}, t \in[0, T]
		\end{equation}
		令Z是与$\mathcal{F}_{+\infty}$独立的随机变量,这里$\mathcal{F}_{+\infty}$是由$B_s(\cdot),s\geq0$,
		且Z满足$E[|Z|^2]<\infty$则随机微分方程
		\begin{equation}
dX_t=b(t,X_t)dt+\sigma(t,X_t)dB_t,   0\leq t\leq T,X_0=Z
\end{equation}
存在唯一解
$X_t(\omega)\in(M^2)^d$
\subsection{证明}
利用不动点定理证明设$\Phi:(M^2)^d\rightarrow(M^2)^d$
$$\forall U\in (M^2)^d,\Phi(U)_t=X+\int_{0}^{t}f(s,U_s)ds+\int_{0}^{t}g(s,U_s)dB_s$$
由(1)(2)知$\Phi(U)\in (M^2)^d$,定义范数$\|U\|_{\beta}=\left(E \int_{0}^{T} e^{-\beta t}\left|X_{t}\right|^{2} d t\right)^{1 / 2}$显然它与$(M^2)^d$的原范数等价下证存在$\beta$
使得$\Phi$为严格压缩映射,设$U,U'\in(M^2)^d$记
$$
\bar{U}=U-U^{\prime}, \bar{f}_{t}=f\left(t, U_{t}\right)-f\left(t, U_{t}^{\prime}\right), \bar{g}_{t}=g\left(t, U_{t}\right)-g\left(t, U_{t}^{\prime}\right), \bar{\Phi}_{t}=\Phi(U)_{t}-\Phi\left(U^{\prime}\right)_{t}
$$
对$\forall \beta \in \mathbb{R}$,对$e^{-\beta t}$|\bar{\Phi_T}|^2$在$[0,T]$上利用ito公式得	
\begin{equation}
\begin{align*}
&e^{-\beta T}\left|\bar{\Phi}_{T}\right|^{2}+\beta \int_{0}^{T} e^{-\beta t}\left|\bar{\Phi}_{t}\right|^{2} d t \\
&=2 \int_{0}^{T} e^{-\beta t}\left\langle\bar{\Phi}_{t}, \bar{f}_{t}\right\rangle d t+2 \int_{0}^{T} e^{-\beta t}\left\langle\bar{\Phi}_{t}, \bar{g}_{t} d B_{t}\right\rangle+\int_{0}^{T} e^{-\beta t} \operatorname{Tr}\left[\bar{g}_{t} \bar{g}_{t}^{*}\right] d t
\end{align*}
\end{equation}	
对上式两边取期望
$$
\beta E \int_{0}^{T} e^{-\beta t}\left|\bar{\Phi}_{t}\right|^{2} d t \leqslant 2 E \int_{0}^{T} e^{-\beta t}\left\langle\bar{\Phi}_{t}, \bar{f}_{t}\right\rangle d t+E \int_{0}^{T} e^{-\beta t} \operatorname{Tr}\left[\bar{g}_{t} \bar{g}_{t}^{*}\right] d t
$$
然后利用(2)
$$
\beta E \int_{0}^{T} e^{-\beta t}\left|\bar{\Phi}_{t}\right|^{2} d t \leqslant E \int_{0}^{T} e^{-\beta t}\left|\bar{\Psi}_{t}\right|^{2} d t+K^{2} E \int_{0}^{T} e^{-\beta t}\left|\bar{U}_{t}\right|^{2} d t+K^{2} E \int_{0}^{T} e^{-\beta t}\left|\bar{U}_{t}\right|^{2} d t
$$
当$\beta=1+4K^2$时,便有
$$
E \int_{0}^{T} e^{-\beta t}\left|\bar{\Phi}_{t}\right|^{2} d t \leqslant \frac{1}{2} E \int_{0}^{T} e^{-\beta t}\left|\bar{U}_{t}\right|^{2} d t
$$
这样即证$\Phi$为严格的压缩映射
\section{ito-Talor展开}
\subsection{引入多重积分}
考虑一维sde
\begin{equation}
X_{t}=X_{t_{0}}+\int_{t_{0}}^{t} a\left(X_{s}\right) d s+\int_{t_{0}}^{t} b\left(X_{s}\right) d W
\end{equation}
这里$t\in[t_0,T]$a,b足够光滑且满足有界线性增长,则对任意二阶连续可微函数由ito公式可得
\begin{equation}
\begin{aligned}
f\left(X_{t}\right)=& f\left(X_{t_{0}}\right) \\
&+\int_{t_{0}}^{t}\left(a\left(X_{s}\right) \frac{\partial}{\partial x} f\left(X_{s}\right)+\frac{1}{2} b^{2}\left(X_{s}\right) \frac{\partial^{2}}{\partial x^{2}} f\left(X_{s}\right)\right) d s \\
&+\int_{t_{0}}^{t} b\left(X_{s}\right) \frac{\partial}{\partial x} f\left(X_{s}\right) d W_{s} \\
=& f\left(X_{t_{0}}\right)+\int_{t_{0}}^{t} L^{0} f\left(X_{s}\right) d s+\int_{t_{0}}^{t} L^{1} f\left(X_{s}\right) d W_{s}
\end{aligned}
\end{equation}
这里$L^0$和$L^1$分别表示
\begin{equation}
L^{0}=a \frac{\partial}{\partial x}+\frac{1}{2} b^{2} \frac{\partial^{2}}{\partial x^{2}}
\end{equation}
\begin{equation}
L^{1}=b \frac{\partial}{\partial x}
\end{equation}
下面令$f=X_t$,且对a和b继续使用ito公式可得
\begin{equation}
\begin{aligned}
X_{t}=& X_{t_{0}} \\
&+\int_{t_{0}}^{t}\left(a\left(X_{t_{0}}\right)+\int_{t_{0}}^{s} L^{0} a\left(X_{z}\right) d z+\int_{t_{0}}^{s} L^{1} a\left(X_{z}\right) d W_{z}\right) d s \\
&+\int_{t_{0}}^{t}\left(b\left(X_{t_{0}}\right)+\int_{t_{0}}^{s} L^{0} b\left(X_{z}\right) d z+\int_{t_{0}}^{s} L^{1} b\left(X_{z}\right) d W_{z}\right) d W \\
=& X_{t_{0}}+a\left(X_{t_{0}}\right) \int_{t_{0}}^{t} d s+b\left(X_{t_{0}}\right) \int_{t_{0}}^{t} d W_{s}+R
\end{aligned}
\end{equation}
这里R表示
$$
\begin{aligned}
R=& \int_{t_{0}}^{t} \int_{t_{0}}^{s} L^{0} a\left(X_{z}\right) d z d s+\int_{t_{0}}^{t} \int_{t_{0}}^{s} L^{1} a\left(X_{z}\right) d W_{z} d s \\
&+\int_{t_{0}}^{t} \int_{t_{0}}^{s} L^{0} b\left(X_{z}\right) d z d W_{s}+\int_{t_{0}}^{t} \int_{t_{0}}^{s} L^{1} b\left(X_{z}\right) d W_{z} d W_{s}
\end{aligned}
$$
也可以对$L^0a,L^1a,L^0b,L^1b$继续使用ito公式,这就是最简单的$Ito-Taylor$展开。从上可知我们需要处理一个多重积分。
\section{多重积分}
\subsection{Multi-indices}
称行向量
$$
\boldsymbol{\alpha}=\left(j_{1}, j_{2}, \dots, j_{l}\right)
$$
其中$$
j_{i} \in\{0,1, \ldots, m\}
$$
为multi-indices,m表示Brown运动的m个分量$j_i$用于表示积分顺序$L^{j_i}$算子的作用顺序。$v$表示长度为0的multi-indices。
\subsection{积分定义}
以下$f$均为右连续适应的随机过程。令$\mathcal{H}_v$表示满足如下性质的$f$全体
$$|f(t,\omega)|<\infty, w.p.1,t\geq 0$$
类似的$\mathcal{H}_0$表示满足如下性质的$f$全体
$$
\int_{0}^{t}|f(s, \omega)| d s<\infty,w.p.1,t\geq 0
$$
$\mathcal{H}_1$表示满足如下性质的函数全体
$$
\int_{0}^{t}|f(s, \omega)|^2 d s<\infty,w.p.1,t\geq 0
$$
对于$j\geq2$,$\mathcal{H}_j=\mathcal{H}_1$,接下来定义$\mathcal{H}_{\alpha}$。\\
设$\rho$和$\tau$是两个满足$0\leq \rho(\omega)\leq \tau(\omega)\leq T,a.s.$的停时。则对于$\boldsymbol{\alpha}=\left(j_{1}, j_{2}, \dots, j_{l}\right)$递归定义多重积分
\begin{equation}
I_{\alpha}[f(\cdot)]_{\rho, \tau}:=\left\{\begin{array}{ll}
f(\tau) & : \quad l=0 \\
\int_{\rho}^{\tau} I_{\alpha-}[f(\cdot)]_{\rho, s} d s & : \quad l \geq 1 \text { and } j_{l}=0 \\
\int_{\rho}^{\tau} I_{\alpha-}[f(\cdot)]_{\rho, s} d W_{s}^{j_{l}} & : l \geq 1 \text { and } j_{l} \geq 1
\end{array}\right.
\end{equation}
$\mathcal{H}_{\alpha}$的元素应满足$I_{\alpha-}[f(\cdot)]_{\rho,} \in \mathcal{H}_{\left(j_{l}\right)}$
\subsection{多重积分之间的关系}
记号:令$I_{\alpha,t}=I_{\alpha}[1]_{0,t}$,$W_t^0=t$,$I_A$表示示性函数。\\
令$j_{1}, \ldots, j_{l} \in\{0,1, \ldots, m\}$, $\alpha=\left(j_{1}, \ldots, j_{l}\right) \in\mathcal{M}$,$l=1,2,3,...$则
\begin{equation}
\begin{aligned}
W_{t}^{j} I_{\alpha, t}&=\sum_{i=0}^{l} I_{\left(j_{1}, \ldots, j_{i}, j, j_{i+1}, \ldots, j_{l}\right), t} \\
&+\sum_{i=1}^{l} I_{\left\{j_{i}=j \neq 0\right\}} I_{\left(j_{1}, \ldots, j_{i-1}, 0, j_{i+1}, \ldots, j_{l}\right), t}
\end{aligned}
\end{equation}
证明:对$I_{(j),t}I_{\alpha,t}$应用ito公式:
\begin{equation}
\begin{aligned}
W_{t}^{j} I_{\alpha, t}=& I_{(j), t} I_{\alpha, t} \\
=& \int_{0}^{t} I_{\alpha, s} d I_{(j), s}+\int_{0}^{t} I_{(j), s} I_{\alpha-, s} d W_{s}^{j_{l}} \\
&+I_{\left\{j_{i}=j \neq 0\right\}} \int_{0}^{t} I_{\alpha-, s} d s &
\end{aligned}
\end{equation}
当$l=1$时命题得证,对于$l>1$继续对$I_{(j),s}I_{\alpha-,s}$使用如上的展开。反复进行上述操作即可证明上述命题。\\
设$\alpha=\left(j_{1}, \ldots, j_{l}\right)$,$j_1=...=j_l=j\in {0,...,m}$,$l\gqe 2$则
\begin{equation}
I_{\alpha, t}=\left\{\begin{array}{ll}
\frac{1}{l!}t^{l} & : \quad j=0 \\
\frac{1}{l}\left(W_{i}^{j} I_{\alpha-, t}-t I_{(\alpha-)-, t}\right): & j \geq 1
\end{array}\right.
\end{equation}
证明$j=0$的情况就是普通的重积分,对于$j\in {1,...m}$
\begin{equation}
t I_{(\alpha-)-, t}=\sum_{i=0}^{l-2} I_{\left(j_{1}, \dots, j_{i}, 0, j_{i+1}, \dots, j_{l-2}\right), t}
\end{equation}
\begin{equation}
W_{t}^{j} I_{\alpha-, t}=l I_{\alpha, t}+\sum_{i=1}^{l-1} I_{\left(j_{1}, \dots, j_{i-1}, 0, j_{i+1}, \dots, j_{l-1}\right), t}
\end{equation}
移项得证上述公式。
\end{document}
