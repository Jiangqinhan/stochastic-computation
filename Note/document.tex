\documentclass{article}%book,report,letter
\usepackage{ctex}
\usepackage{graphicx}
\graphicspath{{figures/}}

\begin{document}
	\section{sde解的存在唯一性}给定T令$b(\cdot, \cdot):[0, T] \times \mathbf{R}^{n} \rightarrow \mathbf{R}^{n}, \sigma(\cdot, \cdot):[0, T] \times \mathbf{R}^{n} \rightarrow \mathbf{R}^{n \times m}$是可测函数满足
		\begin{equation}
		|b(t, x)|+|\sigma(t, x)| \leq C(1+|x|) ; \quad x \in \mathbf{R}^{n}, \quad t \in[0, T]
		\end{equation}
		\begin{equation}
		|b(t, x)-b(t, y)|+|\sigma(t, x)-\sigma(t, y)| \leq D|x-y| ; \quad x, y \in \mathbf{R}^{n}, t \in[0, T]
		\end{equation}
		令Z是与$\mathcal{F}_{+\infty}$独立的随机变量,这里$\mathcal{F}_{+\infty}$是由$B_s(\cdot),s\geq0$,
		且Z满足$E[|Z|^2]<\infty$则随机微分方程
		\begin{equation}
		dX_t=b(t,X_t)dt+\sigma(t,X_t)dB_t,   0\leq t\leq T,X_0=Z
		\end{equation}
		存在唯一连续解$X_t(\omega)$满足以下性质:\\
		$X_t(\omega)$关于$\mathcal{F}^Z_t$是适应的而且
		\begin{equation}
		E\left[\int_{0}^{T}\left|X_{t}\right|^{2} d t\right]<\infty
		\end{equation}
		\subsection{唯一性}
		这里的唯一性是指满足上述条件的连续解是唯一的,接下来用到$ito$等距公式,$C-S$不等式证明唯一性,不妨设$X_t(\omega)$是方程关于Z的解,$\widehat{X}_t(\omega)$是方程关于$\widehat{Z}$的解令$a(s, \omega)=b\left(s, X_{s}\right)-b\left(s, \hat{X}_{s}\right)$ $\gamma(s, \omega)=\sigma\left(s, X_{s}\right)-\sigma\left(s, \widehat{X}_{s}\right)$则
		\begin{equation}
		\begin{array}{l}
		E\left[\left|X_{t}-\widehat{X}_{t}\right|^{2}\right]=E\left[\left(Z-\widehat{Z}+\int_{0}^{t} \operatorname{ads}+\int_{0}^{t} \gamma d B_{s}\right)^{2}\right] \\
		\quad \leq 3 E\left[|Z-\widehat{Z}|^{2}\right]+3 E\left[\left(\int_{0}^{t} \text { ads }\right)^{2}\right]+3 E\left[\left(\int_{0}^{t} \gamma d B_{s}\right)^{2}\right] \\
		\quad \leq 3 E\left[|Z-\widehat{Z}|^{2}\right]+3 t E\left[\int_{0}^{t} a^{2} d s\right]+3 E\left[\int_{0}^{t} \gamma^{2} d s\right] \\
		\quad \leq 3 E\left[|Z-\widehat{Z}|^{2}\right]+3(1+t) D^{2} \int_{0}^{t} E\left[\left|X_{s}-\hat{X}_{s}\right|^{2}\right] d s
		\end{array}
		\end{equation}
		令
		$$
		v(t)=E\left[\left|X_{t}-\hat{X}_{t}\right|^{2}\right] ; \quad 0 \leq t \leq T
		$$
		则上式满足
		\begin{equation}
		\begin{aligned}
		v(t) & \leq F+A \int_{0}^{t} v(s) d s \\
		&  F=3 E\left[|Z-\widehat{Z}|^{2}\right] ,A=3(1+T) D^{2}
		\end{aligned}
		\end{equation}
		根据Gronwall不等式得
		$v(t)\leq exp(At)$
		现在假设$Z=\widehat{Z}$则F=0所以$v(t)=0$所以
		$$
		P\left[\left|X_{t}-\hat{X}_{t}\right|=0 \quad \text { for all } t \in \mathbf{Q} \cap[0, T]\right]=1
		$$
		由$|X_t-\widehat{X}|$的连续性知,sde的解是唯一的
		\subsection{存在性}
		利用不动点定理证明设$\Phi:(M^2)^d\rightarrow(M^2)^d$
		$$\forall U\in (M^2)^d,\Phi(U)_t=X+\int_{0}^{t}f(s,U_s)ds+\int_{0}^{t}g(s,U_s)dB_s$$
		由(1)(2)知$\Phi(U)\in (M^2)^d$,定义范数$\|U\|_{\beta}=\left(E \int_{0}^{T} e^{-\beta t}\left|X_{t}\right|^{2} d t\right)^{1 / 2}$显然它与$(M^2)^d$的原范数等价下证存在$\beta$
		使得$\Phi$为严格压缩映射,设$U,U'\in(M^2)^d$记
		$$
		\bar{U}=U-U^{\prime}, \bar{f}_{t}=f\left(t, U_{t}\right)-f\left(t, U_{t}^{\prime}\right), \bar{g}_{t}=g\left(t, U_{t}\right)-g\left(t, U_{t}^{\prime}\right), \bar{\Phi}_{t}=\Phi(U)_{t}-\Phi\left(U^{\prime}\right)_{t}
		$$
		对$\forall \beta \in \mathbb{R}$,对$e^{-\beta t}$|\bar{\Phi_T}|^2$在$[0,T]$上利用ito公式得
$$
\begin{equation}
& e^{-\beta T}\left|\bar{\Phi}_{T}\right|^{2}+\beta \int_{0}^{T} e^{-\beta t}\left|\bar{\Phi}_{t}\right|^{2} d t \\
=& 2 \int_{0}^{T} e^{-\beta t}\left\langle\bar{\Phi}_{t}, \bar{f}_{t}\right\rangle d t+2 \int_{0}^{T} e^{-\beta t}\left\langle\bar{\Phi}_{t}, \bar{g}_{t} d B_{t}\right\rangle+\int_{0}^{T} e^{-\beta t} \operatorname{Tr}\left[\bar{g}_{t} \bar{g}_{t}^{*}\right] d t
\end{eqyation}
$$
对上式两边取期望
$$
\beta E \int_{0}^{T} e^{-\beta t}\left|\bar{\Phi}_{t}\right|^{2} d t \leqslant 2 E \int_{0}^{T} e^{-\beta t}\left\langle\bar{\Phi}_{t}, \bar{f}_{t}\right\rangle d t+E \int_{0}^{T} e^{-\beta t} \operatorname{Tr}\left[\bar{g}_{t} \bar{g}_{t}^{*}\right] d t
$$
然后利用(2)
$$
\beta E \int_{0}^{T} e^{-\beta t}\left|\bar{\Phi}_{t}\right|^{2} d t \leqslant E \int_{0}^{T} e^{-\beta t}\left|\bar{\Psi}_{t}\right|^{2} d t+K^{2} E \int_{0}^{T} e^{-\beta t}\left|\bar{U}_{t}\right|^{2} d t+K^{2} E \int_{0}^{T} e^{-\beta t}\left|\bar{U}_{t}\right|^{2} d t
$$
当$\beta=1+4K^2$时,便有
$$
E \int_{0}^{T} e^{-\beta t}\left|\bar{\Phi}_{t}\right|^{2} d t \leqslant \frac{1}{2} E \int_{0}^{T} e^{-\beta t}\left|\bar{U}_{t}\right|^{2} d t
$$
这样即证$\Phi$为严格的压缩映射
		
	
	


\end{document}